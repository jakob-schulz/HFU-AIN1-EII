\documentclass[a4paper,11pt,titlepage]{article}

\usepackage{ucs}
% per input encoding kann man Umlaute direkt einsetzten, aber  dann ist man von Font des jeweiligen Rechners abh"angig. Daher mag ich es nicht!
%\usepackage[utf8x]{inputenc}
\usepackage[german,ngerman]{babel}
\usepackage{fontenc}
\usepackage[pdftex]{graphicx}
%\usepackage{latexsym}

\usepackage[pdftex]{hyperref}

\begin{document}

% hier aktuelle Uebungsnummer einfuegen
\title{Einf\"uhrung in die Informatik\\
Ausarbeitung \"Ubung 2}

% Namen der Bearbeiter einfuegen

\author{Jakob Schulz}

% aktuelles Datum einfuegen

\date{\today}

\maketitle{\thispagestyle{plain}}

\section{Darstellung von Umlauten}
Die Darstellung von Umlauten erfolgt durch das Voranstellen von \textbackslash'' und dem betreffenden Selbstlaut oder nur durch das Voranstellen von '' und dem betreffenden Selbstlaut. \\
Beispiel: \\ Aus \textbackslash'' a wird "a und aus ''a wird auch "a


\section{Absätze}
% hier dann die eigene Bearbeitung einfuegen
Ein neuer Absatz kann durch \textbackslash \textbackslash ~oder durch \textbackslash newline gekennzeichnet werden.

\section{Listen}
% hier dann die eigene Bearbeitung einfuegen
Nicht nummerierte Listen lassen sich durch \textbackslash begin itemize angeben und durch \textbackslash end itemize beenden.\\
Nummerierte Listen lassen sich durch \textbackslash begin enumerate angeben und durch \textbackslash begin enumerate beenden.

\section{Tabellen}
Tabellen lassen sich mit \textbackslash begin \{ tabular\} \{\dots\} anlegen und mit \textbackslash end \{ tabular\} beenden.
Hierbei wird eingesetzt, ob der Text in den Tabellen linksbündig (l), rechtsbündig (r) oder zentriert (c) ist. Der Befehl p\{breite\} gibt die Breite einer Spalte mit mehrzeiligem 
Text an. \\
Innerhalb der Tabelle kann man mit \& in die nächste Tabellenspalte springen.

\section{Einfügen von Bildern}
Um Bilder einzufügen muss zuerst im Vorspann des Dokuments das Paket graphicx eingefügt werden.\\
Befehl: \textbackslash usepackage \{graphicx \} \\
Mit \textbackslash includegraphics [width = \dots cm] \{ "`name" ' \} einfügen. \\
Wichtig: Das Bild muss hierbei im selben Ordner wie die LaTeX- Datei sein.
\\
Beispiel: \newpage
\includegraphics [width = 10 cm] {Aufgabe} \\

% ---------
\section{Resumee zur dieser "Ubungsaufgabe}
Dauer f"ur 
\begin{itemize}
	\item Durchf"uhrung: Lässt sich nicht genau sagen, da Durchführung darin bestand die Dokumentation zu Verstehen
	\item Dokumentation: ''
\end{itemize}
Probleme: Einarbeitung in bisher ungewohnte Umgebung und Verstehen der Syntax.
Test f"ur github.

\end{document}
