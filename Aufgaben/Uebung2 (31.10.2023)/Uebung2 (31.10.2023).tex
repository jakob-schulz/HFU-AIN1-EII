\documentclass[a4paper,11pt,titlepage]{article}

\usepackage{ucs}
% per input encoding kann man Umlaute direkt einsetzten, aber  dann ist man von Font des jeweiligen Rechners abh"angig. Daher mag ich es nicht!
%\usepackage[utf8x]{inputenc}
\usepackage[german,ngerman]{babel}
\usepackage{fontenc}
\usepackage[pdftex]{graphicx}
%\usepackage{latexsym}
\usepackage{color}
\usepackage[pdftex]{hyperref}

\begin{document}

% hier aktuelle Uebungsnummer einfuegen
\title{Einf\"uhrung in die Informatik\\
Ausarbeitung \"Ubung 2}

% Namen der Bearbeiter einfuegen

\author{Jakob Schulz, Julian Niethammer}

% aktuelles Datum einfuegen

\date{31. Oktober 2023}

\maketitle{\thispagestyle{plain}}

\section{Umrechnung zwischen Zahlensystemen}

\subsection{Problem}
Zahlen von verschiedenen Zahlensystemen (Binär, Dezimal, Hexadezimal und Octal) umrechnen in die jeweiligen anderen Zahlensysteme.
\subsection{L"osungskonzept}
Die anderen Zahlensysteme entsprechen dem Dezimalsystem. Unterschied besteht darin, dass die Stellen der Zahl eine andere Bedeutung haben.\\
Beim Dezimalsystem steht jede Stelle für ein Vielfaches von 10. Beim Oktalsystem steht wiederrum jede Stelle für ein Vielfaches von 8 und beim Binärsystem dementsprechend jede Stelle für ein Vielfaches von 2\\
Bsp.:\\
Dezimalsystem in Oktalsystem:\\
\scriptsize 1000 100 10 1\qquad\qquad 4096 512 64 8 1 \\
\normalsize 
$4 \quad \ 2\quad3\ 0_{10} =\quad 1 \quad \ 0\quad2\ 0\ 6_8$\\
\\
Somit muss man, wenn man eine Dezimalzahl in ein beliebiges anderes Zahlensystem umrechnen möchte, nur die Basis (Bin"ar: 2, Oktal: 8, ...) mit Rest teilen. Sofern ein Rest vorhanden ist repr"asentiert dieser die niederwertigste "`freie"' Stelle. Das Ergebnis ohne den Rest wird wieder durch die Basis geteilt. Das ganze wiederholt sich solange, bis das Ergebnis ohne den Rest null ist. Das Prinzip besteht somit darin, dass man die Zahl von rechts nach links aufbaut.\\
Bsp.:\\
Dezimalzahl 123456 in Hexadezimalzahl umwandeln\\
\\
\begin{tabular}{lll}
Rechnung&ganzzahliges Ergebnis&Rest\\
$123456 \div 16$ &7716&0\\
$7716 \div 16$ &482&4\\
$482 \div 16$ &30&2\\
$30 \div 16$ &1&E\\
$1 \div 16$ &0&1\\
\end{tabular}\\
\\
Hexadezimalzahl: $1E240_{16}$\\
\\
Will man nun eine Zahl von einem Ausgangssystem in ein Zielsystem umwandeln (keines der Systeme ist Dezimalsystem), bietet es sich an die Zahl vom Ausgangssystem erst in das Dezimalsystem umzuwandeln und dann in das Zielsystem umzuwandeln.\\
Um eine Zahl von einem anderen System in ein Dezimalsystem umzuwandeln geht man wie folgt vor:\\
Man nimmt jeden Repräsentanten einer Stelle der Zahl und multipliziert diesen mit dem Wert der entsprechenden Stelle. Anschließend addiert man alle Ergebnisse zusammen und man hat die Dezimalzahl.\\
Bsp.: $101011= 1\cdot 1 + 1\cdot 2 + 0\cdot 4 +1\cdot 8 +0\cdot 16 + 1\cdot 32 = 43$


\subsection{Beispielaftes Umrechnen von Zahlen}

$192_{10}\ \widehat{=}\ 11000000_2\ \widehat{=}\ C0_{16}\ \widehat{=}\ 300_{8}\\
\\
0C_{16}\ \widehat{=}\ 12_{10}\ \widehat{=}\ 1100_2\ \widehat{=}\ 14_8\\
\\
764_{8}\ \widehat{=}\ 500_{10}\ \widehat{=}\ 111110100_2\ \widehat{=}\ 1F4_{16}\\
\\
01111110_2\ \widehat{=}\ 126_{10}\ \widehat{=}\ 176_8\ \widehat{=}\ 7E_{16}$
\\

\subsection{Tests}
Das Ergebnis lässt sich überprüfen, indem man die Zahlen aus jedem Zahlensystem in eine Dezimalzahl umwandelt. Sind alle Dezimalzahlen gleich, geben die Zahlen der anderen Systeme alle die gleiche Zahl wieder.

\subsection{Beispiel aus der Thematik IPv4 Adressen}
\subsubsection{Aufgabe}
Den unteren und oberen darstellbaren Wert im Dezimalsystem und im Hexadezimalsystem bestimmen
\subsubsection{Ausarbeitung}
\begin{itemize} 
\item Sie haben 8 Bit zur Informationsdarstellung: Die Wertedarstellung geht von $0_{10} = 00_{16}$ 
bis $255_{10} = FF_{16}$
\item Das h"ochstwertige Bit muss 0 sein: Die Wertedarstellung geht von $0_{10} = 00_{16}$ 
bis $127_{10} = 7F_{16}$
\item Jetzt muss das h"ochstwertige Bit immer 1 sein, das zweith"ochste Bit muss 0 sein: 
Die Wertedarstellung geht von: $128_{10} = 80_{16}$ bis $191_{10} =BF_{16}$
\item  Jetzt mussen das h"ochste und das zweith"ochstes Bit 1 gesetzt sein, das dritth"ochste Bit muss 0 sein: 
Die Wertedarstellung geht von $192_{10} = C0_{16}$ bis $223_{10} =DF_{16}$
\end{itemize}


\section{Gebrochenrationale Zahlen}
\subsection{Gr"oßtm"ogliche Zahl berechnen}
\subsubsection{Aufgabe}
Gr"oßt m"ogliche Dezimalzahl mit 4 Bit Vor- und Nachkommastellen berechnen.
\subsubsection{Anastz}
$1111.1111_2 =  1\cdot 1 + 1\cdot 2 + 1\cdot 4 +1\cdot 8 + 1\cdot\frac{1}{2} + 
1\cdot\frac{1}{4} + 1\cdot\frac{1}{8}+1\cdot\frac{1}{16} = 15.9375$
\subsection{Lücken in der Tabelle füllen}
\begin{tabular}{|c|c|c|}
\hline
Dualsystem&Oktalsystem&Hexadezimalsystem\\
\hline
101101.101&\textcolor{blue}{55.5}&\textcolor{blue}{2D.A}\\
\hline
\textcolor{blue}{10101011.11001101}&\textcolor{blue}{253.632}&AB.CD\\
\hline
\end{tabular}\\
\\
Eingesetzte Werte sind blau.
\section{Bin"are Addition/Subtraktion}
\subsection{Addition im Dualsystem}
\subsubsection{Aufgaben}
Aufgaben lösen, indem man Dezimalzahlen zuerst in Dualsystem umwandelt und in diesem dann rechnet
\subsubsection{Ansatz}

\begin{tabular}{lrrc}
$125_{10} + 199_{10} = $&&$01111101\ $\\
&+&$11000111\ $\\
\cline{3-3}
&=&$101000100_2$ &$=324_{10}$\\
\end{tabular}\\
\noindent$27_{10} + 30_{10} = 11011_2 + 11110_2 = 111001_2 = 57_{10}\\
115_{10} + 21_{10} = 1110011_2 + 10101_2 = 10001000_2 = 136_{10}$
\subsection{Subtraktion im Dualsystem}
\subsubsection{Aufgabe}
8 stellige Binärzahlen unter Verwendung des Zweierkomplements subtrahieren.
\subsubsection{Ansatz}
\begin{tabular}{lrrc}
$55_{10} - 120_{10} = $&&$00110111\ \ $\\
&+&$10001000\ \ $\\
\cline{3-3}
&=&$10111111_Z$ &$= -65_{10}$\\
\end{tabular}\\
\\
$42_{10} - 12_{10} = 00101010 + 11110100 = 00011110_Z = 30_{10}\\
18_{10} - 105_{10} = 00010010 + 10010111 = 10101001_Z = -87_{10}$\\
Z steht f"ur Zweierkomplement
\subsection{Warum ist die Begrenzung auf 8 Bin"arstellen zwingend notwendig?}
Ohne die Bergenzung auf 8 Bin"arstellen kann beim h"ochstwertigen Bit aufgrund des "Ubertrages ein neues h"öchstwertiges Bit entstehen. Dadrch w"urde eine falsche Zahl dargestellt werden.\\
Bsp f"ur unbegrenzte Bin"arstellen:\\
\begin{tabular}{lrlc}
$42_{10} - 12_{10} = $&&$\ 00101010 $\\
&+&$\ 11110100$\\
&&\footnotesize111\\
\cline{3-3}
&=&$100011110_Z$ &$\neq 30_{10}$\\
\end{tabular}\\
\section{Resumee zur dieser "Ubungsaufgabe}
Dauer f"ur 
\begin{itemize}
	\item Durchf"uhrung: ca. 1 Stunde
	\item Dokumentation: ca. 3 Stunden
\end{itemize}
Welche gro"sen Probleme waren zu l"osen?\\
Einarbeitung in die Umgebung von LaTex und anschauliche Darstellung des L"osungsansatzes

\end{document}
